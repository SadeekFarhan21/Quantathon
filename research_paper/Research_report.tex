\documentclass[10pt]{article}

\usepackage{libertine}
\usepackage[T1]{fontenc}
\usepackage[utf8]{inputenc}
\usepackage{microtype}

\usepackage{amsmath}
\usepackage{graphicx}
\usepackage{geometry}
\usepackage{fancyhdr}
\usepackage{natbib}
\usepackage{xcolor}
\usepackage{titlesec}
\usepackage{booktabs}
\usepackage[colorlinks=true,linkcolor=black,citecolor=black,urlcolor=black]{hyperref}

\geometry{
    margin=1in,
    headheight=12pt
}

\pagestyle{fancy}
\fancyhf{}
\renewcommand{\headrulewidth}{0.4pt}
\renewcommand{\footrulewidth}{0.4pt}
\fancyhead[L]{\small 2025 OSU Hackathon}
\fancyhead[R]{\small\thepage}
\fancyfoot[C]{\small Extended Abstract}

\titleformat{\section}
    {\normalfont\large\bfseries}{\thesection}{1em}{}
\titlespacing*{\section}
    {0pt}{2.5ex plus 1ex minus .2ex}{1.5ex plus .2ex}

\makeatletter
\renewcommand{\maketitle}{%
    \begin{center}
        \vspace*{0.5cm}
        \Large\@title
        
        \vspace{0.4cm}
        \large\@author
        
        \vspace{0.5cm}
        \normalsize\textit{Quantathon 2025}
        
        \vspace{0.3cm}
        \normalsize Extended Abstract
        \vspace{0.5cm}
    \end{center}
}
\makeatother

% Paper info
\title{Title of Your Paper}

% Uncomment and update the following line upon acceptance
\author{Jalen Francis, Aditya Bhati, Farhan Sadeek, Jayson Clark, and Andrew McKenzie}

\date{03/02/2025}

\begin{document}

\maketitle

\begin{abstract}
This document presents a comprehensive quantitative system for predicting market states and providing optimal investment strategies. The system classifies market periods as Bull (20\% or more increase in stock prices), Bear (20\% or more decline in stock prices), or Static based on drawdown metrics, then employs machine learning and deep learning models to predict future states. The prediction results are used to implement multiple investment strategies that dynamically allocate between equities and bonds. Additionally, the system incorporates advanced anomaly detection, catastrophe modeling, and tail risk analysis to enhance decision-making during extreme market events such as COVID-19. Backtesting demonstrates that the prediction-based strategy achieves comparable or even improved returns to buy-and-hold with significantly reduced drawdown risk and higher risk-adjusted returns.
\end{abstract}

\section{Problem Statement}
This section introduces the problem, provides background information, and states the paper's objectives. Keep it brief, as this is a short paper format.

\section{Analysis}
Describe your methodology here. This section should include details of the approach, techniques, and processes used in the study.

\section{EDA}
Present the results here. This section should summarize the findings without going into extensive detail, appropriate for a 2-page limit.

\section{Approach}
Conclude the paper by summarizing the findings, implications, and potential future work.

\section{Results}

\section{Performance}

\section{Conclusion}

\bibliographystyle{unsrt}
\bibliography{main}

\end{document}